\documentclass[10pt]{beamer}

\usetheme[progressbar=frametitle]{metropolis}
\usepackage{appendixnumberbeamer}

\usepackage{booktabs}
\usepackage[scale=2]{ccicons}

\usepackage{pgfplots}
\usepgfplotslibrary{dateplot}
\usepackage{listings}
\usepackage{parcolumns}


\usepackage{xspace}
\newcommand{\themename}{\textbf{\textsc{metropolis}}\xspace}

\title{Implementing your  SasModels in Futhark}
\subtitle{It goes much faster}
% \date{\today}
\date{}
\author{Mikkel Storgaard Knudsen}
% \institute{Center for modern beamer themes}
% \titlegraphic{\hfill\includegraphics[height=1.5cm]{logo.pdf}}

\begin{document}

\maketitle

\section{Main reasoning behind writing models in Futhark}
\begin{frame}[fragile]{What languages can we model in?}
  \begin{itemize}
  \item Current sasview models are implemented in either pure Python (a programming language)
  \item .. or in OpenCL (much faster than Python)
  \item But should be written in Futhark (much faster than OpenCL!)
  \end{itemize}
  % vis billede af noget diff
\end{frame}
\begin{frame}[fragile]{Why is Futhark faster?}
  \begin{itemize}
  \item Independent (accumulating) for-loop iterations can easily* be rewritten as
    as map/reduce operations.

  \item The Futhark language uses a heavily optimizing compiler, that rewards
    map/reduce patterns with major performance increases.

  \item (In case of basic Python implemented models, it's trivial to make them go
    faster.)
  \end{itemize}
  % vis billede af noget diff
\end{frame}

\section{Porting an OpenCL model to Futhark}
\section{Performance}
\begin{frame}{Average eval time for Line (trivial Python model)}
  \begin{figure}
    \begin{tikzpicture}
      \begin{axis}[
        mbarplot,
        title={1D performance},
        xlabel={nq},
        ylabel={Time in milliseconds},
        width=0.9\textwidth,
        height=3.5cm,
        symbolic x coords={100,500,2500,5000,10000},
        bar width=9pt,
        enlargelimits=0.15,
        ybar=10pt,% configures ‘bar shift’
        bar width=12pt,
        nodes near coords,
        legend style={legend pos=north west}
      ]
      \addplot plot coordinates {(100, 0.24 ) (500, 0.24 ) (2500, 0.26 ) (5000, 0.26 ) (10000, 0.28 )};
      \addplot plot coordinates {(100, 0.93 ) (500, 0.94 ) (2500, 0.97 ) (5000, 0.95 ) (10000, 0.97 )};

      \legend{Current, Futhark}

      \end{axis}
    \end{tikzpicture}
    \begin{tikzpicture}
      \begin{axis}[
        mbarplot,
        title={2D performance},
        xlabel={nq},
        ylabel={Time in milliseconds},
        width=0.9\textwidth,
        height=4cm,
        symbolic x coords={100,500,2500,5000,10000},
        bar width=9pt,
        enlargelimits=0.15,
        ybar=10pt,% configures ‘bar shift’
        bar width=12pt,
        nodes near coords,
        legend style={legend pos=north west}
      ]
      \addplot plot coordinates {(100, 0.31 ) (500, 2.78 ) (2500, 127.01 ) (5000, 501.31 ) (10000, 1970.14 )};
      \addplot plot coordinates {(100, 0.98 ) (500, 1.91 ) (2500, 22.73 ) (5000, 86.03 ) (10000, 338.10 )};

      \legend{Current, Futhark}

      \end{axis}
    \end{tikzpicture}
  \end{figure}
\end{frame}
\begin{frame}{Average eval time for Broad Peak (Python model)}
  \begin{figure}
    \begin{tikzpicture}
      \begin{axis}[
        mbarplot,
        title={1D performance},
        xlabel={nq},
        ylabel={Time in milliseconds},
        width=0.9\textwidth,
        height=3.5cm,
        symbolic x coords={100,500,2500,5000,10000},
        bar width=9pt,
        enlargelimits=0.15,
        ybar=10pt,% configures ‘bar shift’
        bar width=12pt,
        nodes near coords,
        legend style={legend pos=north west}
      ]
      \addplot plot coordinates {(100,0.27) (500,0.30) (2500,0.45) (5000,0.63) (10000,1.22)};
      \addplot plot coordinates {(100,1.22) (500,1.23) (2500,1.23) (5000,1.25) (10000,1.27)};

      \legend{Current, Futhark}

      \end{axis}
    \end{tikzpicture}
    \begin{tikzpicture}
      \begin{axis}[
        mbarplot,
        title={2D performance},
        xlabel={nq},
        ylabel={Time in milliseconds},
        width=0.9\textwidth,
        height=4cm,
        symbolic x coords={100,500,2500,5000,10000},
        bar width=9pt,
        enlargelimits=0.15,
        ybar=10pt,% configures ‘bar shift’
        bar width=12pt,
        nodes near coords,
        legend style={legend pos=north west}
      ]
      \addplot plot coordinates {(100,1.06 ) (500,21.50 ) (2500,675.12) (5000,2686.44 ) (10000,10687)};
      \addplot plot coordinates {(100,1.27 ) (500,2.27 ) (2500,27.48) (5000,103.66 ) (10000,409)};

      \legend{Current, Futhark}

      \end{axis}
    \end{tikzpicture}
  \end{figure}
\end{frame}
\begin{frame}{Average eval time for DAB (trivial OpenCL model)}
  \begin{figure}
    \begin{tikzpicture}
      \begin{axis}[
        mbarplot,
        title={1D performance},
        xlabel={nq},
        ylabel={Time in milliseconds},
        width=0.9\textwidth,
        height=3.5cm,
        symbolic x coords={100,500,2500,5000,10000},
        bar width=9pt,
        enlargelimits=0.15,
        ybar=10pt,% configures ‘bar shift’
        bar width=12pt,
        nodes near coords,
        legend style={legend pos=north west}
      ]
      \addplot plot coordinates {(100, 0.54 ) (500, 0.74 ) (2500, 0.76 ) (5000, 0.74 ) (10000, 0.77 )};
      \addplot plot coordinates {(100, 0.86 ) (500, 0.85 ) (2500, 0.87 ) (5000, 0.87 ) (10000, 0.90 )};

      \legend{Current, Futhark}

      \end{axis}
    \end{tikzpicture}
    \begin{tikzpicture}
      \begin{axis}[
        mbarplot,
        title={2D performance},
        xlabel={nq},
        ylabel={Time in milliseconds},
        width=0.9\textwidth,
        height=4cm,
        symbolic x coords={100,500,2500,5000,10000},
        bar width=9pt,
        enlargelimits=0.15,
        ybar=10pt,% configures ‘bar shift’
        bar width=12pt,
        nodes near coords,
        legend style={legend pos=north west}
      ]
      \addplot plot coordinates {(100, 0.78 ) (500, 1.61 ) (2500, 50.36 ) (5000, 196.19 ) (10000, 771.43 )};
      \addplot plot coordinates {(100, 0.90 ) (500, 1.75 ) (2500, 26.25 ) (5000, 101.59 ) (10000, 402.69 )};

      \legend{Current, Futhark}

      \end{axis}
    \end{tikzpicture}
  \end{figure}
\end{frame}
\begin{frame}{Average eval time for core shell parallelepiped (OpenCL model)}
  \begin{figure}
    \begin{tikzpicture}
      \begin{axis}[
        mbarplot,
        title={1D performance},
        xlabel={nq},
        ylabel={Time in milliseconds},
        width=0.9\textwidth,
        height=3.5cm,
        symbolic x coords={100,500,2500,5000,10000},
        bar width=9pt,
        enlargelimits=0.15,
        ybar=10pt,% configures ‘bar shift’
        bar width=12pt,
        nodes near coords,
        legend style={legend pos=north west}
      ]
      \addplot plot coordinates {(100, 20.88 ) (500, 21.19 ) (2500, 21.99) (5000, 22.44 ) (10000, 42.99)};
      \addplot plot coordinates {(100, 2.68 ) (500, 3.67 ) (2500, 8.00 ) (5000, 13.18 ) (10000,23.90 )};

      \legend{Current, Futhark}

      \end{axis}
    \end{tikzpicture}
    \begin{tikzpicture}
      \begin{axis}[
        mbarplot,
        title={2D performance},
        xlabel={nq},
        ylabel={Time in milliseconds},
        width=0.9\textwidth,
        height=4cm,
        symbolic x coords={100,500,2500,5000,10000},
        bar width=9pt,
        enlargelimits=0.15,
        ybar=10pt,% configures ‘bar shift’
        bar width=12pt,
        nodes near coords,
        legend style={legend pos=north west}
      ]
      \addplot plot coordinates {(100, 0.83 ) (500, 2.17 ) (2500, 61.66 ) (5000, 232.32 ) (10000, 986.73 )};
      \addplot plot coordinates {(100, 2.04 ) (500, 2.88 ) (2500, 23.83 ) (5000, 91.87 ) (10000, 357.89 )};

      \legend{Current, Futhark}

      \end{axis}
    \end{tikzpicture}
  \end{figure}
\end{frame}
\section{Questions?}
\end{document}

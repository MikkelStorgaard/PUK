\documentclass[11pt]{article}
\usepackage[a4paper, hmargin={2.8cm, 2.8cm}, vmargin={2.5cm, 2.5cm}]{geometry}
\usepackage{eso-pic} % \AddToShipoutPicture
\usepackage{graphicx} % \includegraphics

%% Change `ku-farve` to `nat-farve` to use SCIENCE's old colors or
%% `natbio-farve` to use SCIENCE's new colors and logo.
\def \ColourPDF {include/ku-farve}

%% Change `ku-en` to `nat-en` to use the `Faculty of Science` header
\def \TitlePDF   {include/ku-en}  % University of Copenhagen

\title{
  \vspace{3cm}
  \Huge{Optimizing SasView models} \\
  \Large{Enhancing performance by rewriting kernels from OpenCL/OpenMP to Futhark}
}

\author{
  \Large{Mikkel Storgaard Knudsen}
  \\ \texttt{mikkelstorgaard@gmail.com} \\
  \\
  \Large{Supervisor}
  \\ Martin Elsman
  \\ \texttt{mael@di.ku.dk} \\
}

\date{
    \today
}

\begin{document}


\AddToShipoutPicture*{\put(0,0){\includegraphics*[viewport=0 0 700 600]{\ColourPDF}}}
\AddToShipoutPicture*{\put(0,602){\includegraphics*[viewport=0 600 700 1600]{\ColourPDF}}}

\AddToShipoutPicture*{\put(0,0){\includegraphics*{\TitlePDF}}}

\clearpage\maketitle
\thispagestyle{empty}

\section*{Abstract}

\newpage

\section{Introduction}
\subsection{Goal and motivation}
\textit{SasView} is a Small Angle Scattering (SAS) analysis package for the analysis of
1D and 2D scattering data directly in inverse space.
As \textit{SasView} is given data, it performs its analyses by running the input data
through theory models provided by the standalone module \textit{Sasmodels}.
\\
\\
The \textit{Sasmodels} models utilizes a set of kernels, which are implemented
in both OpenCL and OpenMP.
However, benchmark
comparisons\footnote{https://futhark-lang.org/performance.html} between the
parallel algorithm library \textit{Thrust} (implemented in C++/CUDA), and the
array programming language Futhark, shows that there might severe speed
improvements to gain by running these model calculations in Futhark, instead
of in OpenCL and OpenMP.
\\
\\
The goal of this project is to explore the potential speed gains obtained by
rewriting, and testing the resulting speed improvements of, a subset of the
\textit{Sasmodels} models.
The subset will be chosen so that there are at least
one ``simple'' model, one model of intermediate complexity, and one complex
model in it.
\\
\\
If rewriting the models results in a significant\footnote{I.e. a speed increase
  in order of at least a magnitude}, speedup, it will reasonable to believe that the
rest of the models in \textit{Sasmodels} can be rewritten for a similar speedup.

\section{Implementation}
% current architecture of Sasmodels engine
% description of kernelcl kernels
% (smaller) description of kernelcl kernels
% description of current kernelfut implementation
% designovervejelser, hvorfor selvstændige moduler, hvorfor ikke bare udvide kernelpy

\section{Performance tests}
% description of built-in sasmodels comparison functionality
%% description of how comparison module has dictated the design of the futhark kernel (i.e. loading as much data as possible on init)
% display results
% discussion of futhark vs. python performance
% discussion of futhark vs. opencl performance

\section{Discussion, future work}
% is this immediately useful? not on single fire runs with low nq
% 

\section{conclusion}

\end{document}